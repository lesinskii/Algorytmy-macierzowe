
\documentclass{article}
\usepackage[utf8]{inputenc}
\usepackage[polish]{babel}
\usepackage[T1]{fontenc}
\usepackage{amsmath}
\usepackage{amssymb}
\usepackage{lmodern}
\usepackage{graphicx}
\usepackage{algorithm}
\usepackage{algpseudocode}
\usepackage{amsfonts}
\usepackage{float}

\title{Algorytmy macierzowe - rekurencyjne mnożenie macierzy}
\author{Jakub Karoń, Kamil Lesiński \\\\ Grupa 6}
\date{15.10.2024}

\begin{document}

\maketitle

\section{Opis ćwiczenia}
    Proszę wybrać ulubiony język programowania, wygenerować
    macierze losowe o wartościach z przedziału otwartego
    (0.00000001, 1.0) i zaimplementować
    \begin{enumerate}
        \item Rekurencyjne mnożenie macierzy metodą Binét’a (10 punktów)
        \item Rekurencyjne mnożenie macierzy metodą Strassena (10 punktów)
        \item Mnożenie macierzy metodą AI na podstawie artykułu w Nature (10 punktów)
    \end{enumerate}
    Proszę zliczać liczbę operacji zmienno-przecinkowych \verb|(+-*/_liczb_)| wykonywanych podczas mnożenia macierzy.

\section{Środowisko}
    Ćwiczenie zostało wykonane w języku Python. Do obliczeń użyliśmy biblioteki numpy. Do rysowania wykresów użyliśmy bibliotekę matplotlib.

\section{Implementacja algorytmów}
    \subsection{Rekurencyjne mnożenie macierzy metodą Binete'a}

        \textbf{binet\_multiply(A,B):} \hfill \# Binet Matrix Multiplication

        \textbf{Jeżeli} A oraz B mają rozmiar 1:
        \begin{itemize}
            \item \textbf{Zwróć} A*B
        \end{itemize}

        \textbf{W przeciwnym wypadku:}
        \begin{itemize}
            \item Podziel A i B na 4 równych rozmiarów mniejsze macierze
        \end{itemize}

        \[
        A = 
        \begin{bmatrix}
            A_{11} & A_{12} \\
            A_{21} & A_{22}
        \end{bmatrix}
        \]

        \textbf{Zapisz do pomocniczych zmiennych M:}
        \begin{align*}
        M_0 & = binet\_multiply(A_{11}, B_{11}) \\
        M_1 & = binet\_multiply(A_{12}, B_{21}) \\
        M_2 & = binet\_multiply(A_{21}, B_{11}) \\
        M_3 & = binet\_multiply(A_{22}, B_{21}) \\
        M_4 & = binet\_multiply(A_{11}, B_{12}) \\
        M_5 & = binet\_multiply(A_{12}, B_{22}) \\
        M_6 & = binet\_multiply(A_{21}, B_{12}) \\
        M_7 & = binet\_multiply(A_{22}, B_{22})
        \end{align*}

        \textbf{Zapisz macierz C jako:}
        \begin{align*}
        C_1 & = M_0 + M_1 \\
        C_2 & = M_4 + M_5 \\
        C_3 & = M_2 + M_3 \\
        C_4 & = M_6 + M_7
        \end{align*}

        \textbf{Zwróć C}

    \subsection{Rekurencyjne mnożenie macierzy metodą Strassena}
        \textbf{strassen\_multiply(A,B):} \hfill \# Strassen Matrix Multiplication

        \textbf{Jeżeli} A oraz B mają rozmiar 1:
        \begin{itemize}
            \item \textbf{Zwróć} A*B
        \end{itemize}
        
        \textbf{W przeciwnym wypadku:}
        \begin{itemize}
            \item Podziel A i B na 4 równych rozmiarów mniejsze macierze
        \end{itemize}
        
        \[
        A = 
        \begin{bmatrix}
            A_{11} & A_{12} \\
            A_{21} & A_{22}
        \end{bmatrix}
        \]
        
        \textbf{Zapisz do pomocniczych zmiennych M:}
        \begin{align*}
        M_1 &= strassen\_multiply(A_{11} + A_{22}, B_{11} + B_{22}) \\
        M_2 &= strassen\_multiply(A_{21} + A_{22}, B_{11}) \\
        M_3 &= strassen\_multiply(A_{11}, B_{12} - B_{22}) \\
        M_4 &= strassen\_multiply(A_{22}, B_{21} - B_{11}) \\
        M_5 &= strassen\_multiply(A_{11} + A_{12}, B_{22}) \\
        M_6 &= strassen\_multiply(A_{21} - A_{11}, B_{11} + B_{12}) \\
        M_7 &= strassen\_multiply(A_{12} - A_{22}, B_{21} + B_{22})
        \end{align*}
        
        \textbf{Zapisz macierz C jako:}
        \begin{align*}
        C_1 &= M_1 + M_4 - M_5 + M_7 \\
        C_2 &= M_3 + M_5 \\
        C_3 &= M_2 + M_4 \\
        C_4 &= M_1 - M_2 + M_3 + M_6
        \end{align*}
        
        \textbf{Zwróć C}
    
    \subsection{Rekurencyjne mnożenie macierzy zaproponowane przez AI}
\section{Analiza pomiarów}
\end{document}
